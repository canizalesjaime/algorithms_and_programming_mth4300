\documentclass[a4paper]{article}

\usepackage{fullpage} % Package to use full page
\usepackage{parskip} % Package to tweak paragraph skipping
\usepackage{tikz} % Package for drawing
\usepackage{amsmath}
\usepackage{hyperref}

\title{MTH 4300/4299: Algorithms, Computers, and Programming II}
\author{HW \#1}
\date{Due Date: September 14th, 2025}

\begin{document}
\maketitle

\section*{Instructions}
You will need to submit two files to brightspace, a screenshot and a cpp file named: name\_hw1.cpp.
\\\\
For the cpp file, each problem below(2-4) should be written as a function, that is called in your main function. 


\section{}
Submit a screenshot of you compiling and running any of the problems below on your local computer(2-4)

\section{}
Write code that calculates max\{a,b\} and stores the result in the variable m. 


\section{}
Write a code that checks whether the real number x belongs to the union of the open intervals 
(5 - 15)U(95 - 202). If it does, the program should print yes. If it does not, the program should print no.


\section{}
Assume that the user is asked to provide an integer x through the standard input.
The program will check whether x is bigger than 100. If it is, then the program
will print: "Congratulations! You know about big numbers!" and the program will then exit.
If x is not bigger than 100, then the program will print:"Try again with a bigger number:"
and then the user should be prompted to enter another number. The program can only end once
the user has entered a number larger than 100, and should repeatedly prompt the user to enter 
another number if the number provided is smaller than 100.

\end{document}